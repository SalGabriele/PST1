\documentclass{VUMIFPSkursinis}
\usepackage{algorithmicx}
\usepackage{algorithm}
\usepackage{algpseudocode}
\usepackage{amsfonts}
\usepackage{amsmath}
\usepackage{booktabs}
\usepackage{blindtext}
\usepackage{bm}
\usepackage{caption}
\usepackage{color}
\usepackage{colortbl}
\usepackage{float}
\usepackage{graphicx}
\usepackage{listings}
\usepackage{multirow}
\usepackage{scrextend}
\usepackage{subfig}
\usepackage{wrapfig}
\usepackage{longtable}
\usepackage{enumitem}
\usepackage{xparse}
%\usepackage{tabularx}
\usepackage{ltxtable}
\usepackage{tabu}
\usepackage{xcolor}

% Titulinio aprašas
\university{Vilniaus universitetas}
\faculty{Matematikos ir informatikos fakultetas}
\department{Programų sistemų katedra} %instituuuuta
\papertype{Testavimo atvejai ir rasti defektai}
\title{Programos ,,Christmas Time Calculator" testavimas}
\status{3 kurso 5 grupės studentė}
\author{Gabrielė Saletytė}
\supervisor{dr. Vytautas Valaitis}
\date{Vilnius – \the\year}

% Nustatymai
% \setmainfont{Palemonas}   % Pakeisti teksto šriftą į Palemonas (turi būti įdiegtas sistemoje)
%\bibliography{bibliography}
\input{customCommands}

%==========================================================================> Dokumento pradžia <========================================================================
\begin{document}
    \newenvironment{innerParagraph}[1][0.5cm]{\begin{adjustwidth}{#1}{}}{\end{adjustwidth}}
        
    \pagenumbering{gobble}
    \maketitle
    \tableofcontents
	\pagenumbering{arabic}
	
	\sectionnonum{Anotacija} \label{anotacija}
		Šiame laboratoriniame darbe bus testuojama „Christmas Time Calculator“ programa.
		Ši programėlė skirstoma į dvi dalis - pagrindinę, kurią sudaro laiko iki Kalėdų skaičiuoklė, bei papildomą - 
		kurią sudaro kalėdinis žaidimas.
		Sudarant testavimo atvejus bus atsižvelgtą į atskirus dalių funkcionalumus.
		Ši programa rašyta 2016 m. rugsėjo - gruodžio mėn. ir atsiskaityta kaip taikomojo objektinio programavimo užduotis.
		Programa rašyta C# kalba.

		Pradžioje bus apibrėžti reikalavimai programai, pagal kuriuos buvo atliekama užduotis.
		Vėliau bus pateikiami testavimo atvejai ir sudaryta atsekamumo matrica.
		Laboratorinio darbo pabaigoje bus pateiktas defektų sąrašas ir sudarytas terminų žodynėlis.
	\sectionnonum{Įvadas} \label{ivadas}
		\subsection*{Testuojamos programos pavadinimas} \label{ivadas_psPavadinimas}
			,,Christmas Time Calculator''.
		\subsection*{Darbo tikslas} \label{ivadas_darboTikslas}
			Panaudojant testavimo atvejų sudarymo strategiją rankiniu būdu, 
			ištestuoti „Christmas Time Calculator“ programą, sudaryti testavimo atvejus bei defektų sąrašus.
		\subsection*{Darbo pagrindas} \label{ivadas_pagrindas}
			Dokumentas parengtas kaip programų sistemų testavimo laboratorinis darbas.
	
	\section{Reikalavimai programai ,,Christmas Time Calculator"} \label{reikalavimai}
		Šiame skyriuje pateikiami reikalavimai, į kuriuos buvo atsižvelgiama rašant „Christmas Time Calculator“ programą.
		Reikalavimai pateikiami bendrai, neišskiriant atskirų programos dalių funkcionalumų.
		\begin{enumerate}[label=\textbf{R\arabic*}]
			\item Pagrindiniame programos lange turi būti galimybė keisti „Until Christmas Left“ teksto spalvą paspaudus ant šio užrašo.
			Pasirinkta spalva turi išlikti išjungus programą.
			\item Pagrindiniame lange pasikeitus mėnesių, dienų, valandų, minučių ar sekundžių skaičiui į 1, 
			šalia esantis žodis iš daugiskaitos turi
			pasikeisti į vienaskaitą.
			\item Likus kiekvienai minutei mažiau (t. y. programa rodo 0 ties sekundžių skiltimi), naujame lange įsijungia kalėdinis žaidimas.
			\item Keičiant pagrindinės dalies dydį, visi laukai proporcingai pakeičia savo dydį. 
			Kalėdinio žaidimo lango dydžio keisti nėra galimybės.
			\item Esant įjungtam kalėdiniam žaidimui ir vėl atėjus laikui įjungti jį, žaidimas neįsijungia iš naujo ir 
			nėra atidaromas papildomas langas su nauju žaidimu.
			\item Žaidimo metu su kepurėle pagavus dovanėlę taškai padidėja vienu vienetu.
			Pasiekus naują rekordinį taškų kiekį, skaičius ties „High Score“ nuolat atnaujinamas į dabar surinktų taškų kiekį.
			\item Kalėdinės kepurėlės judėjimas pagrįstas rodyklių bei klavišų „A“ ir „D“ valdymu.
			\item Dovanėlei nukritus iš matomo lauko, prarandama viena gyvybė tai pašalinus vieną iš 3 dovanėlių ikonų.
			\item Krentančios kalėdinės dovanėlės turi būti pilnai matomos žaidimo erdvėje. 
			Dovanėlės negali kristi už matymo ribų ir negali matytis tik dalinai.
			\item Ilgėjant žaidimo laikui, dovanėlių bei kalėdinės kepurėlės judėjimo greičiai proporcingai didėja.
			\item Praradus visas gyvybes, žaidimas turi išsijungti po 5 sekundžių.
			Prieš išsijungiant ekrane turi pasirodyti „Game Over“ užrašas.
			Jei žaidimo metu pasiektas naujas rekordas, vietoj „Game Over“ užrašo turi pasirodyti „New High Score!“ užrašas. 
		\end{enumerate}
	\section{Programos ,,Christmas Time Calculator" testavimas} \label{testavimas}
		Šiame skyriuje bus pateikiami testavimo atvejai bei reikalavimų - testavio atvejų atsekamumo matrica.
		\subsection{Testavimo atvejai} \label{testavimoAtvejai}
			Šioje dalyje pateikiamos lentelės su atskirais testavimo atvejais.
			\subsubsection{Pagrindinės dalies testavimo atvejai} \label{pagrindinesDaliesTA}
				Toliau pateikiami testavimo atvejai, apimantys tik pagrindinės dalies funkcionalumą.
			\subsubsection{Papildomos dalies testavimo atvejai} \label{papildomosDaliesTA}
				Toliau pateikiami testavimo atvejai, apimantys tik papildomos dalies („Christmas Game“) funkcionalumą.
		\subsection{Reikalavimų - testavimo atvejų atsekamumo matrica} \label{atsekamumo matrica}
			Toliau pateikiama reikalavimų - testavomo atvejų atsekamumo matrica.
	\section{Defektų sąrašas} \label{defektai}

	\sectionnonum{Išvados} \label{isvados}
	\sectionnonum{Rezultatai} \label{rezultatai}
\end{document}